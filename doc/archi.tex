%%%% OPTION
%% Change class according to your needs
%%  - article (no chapter)
%%  - report
%%  - etc.
\documentclass{report}


\IfFileExists{ifxetex.sty}{%
  \usepackage{ifxetex}%
}{%
  \newif\ifxetex
  \xetexfalse
}
  \ifxetex

\usepackage{fontspec}
\usepackage{xltxtra}
\setmainfont{DejaVu Serif}
\setsansfont{DejaVu Sans}
\setmonofont{DejaVu Sans Mono}
\else
\usepackage[T1]{fontenc}
\usepackage[latin1]{inputenc}
\fi
\usepackage{fancybox}
\usepackage{makeidx}
\usepackage{cmap}
\usepackage{url}
\usepackage{eurosym}
\usepackage{graphicx}

\usepackage[hyperlink]{sysfera}



%%%%
%% TODO:
%%  - ajouter une macro pour mettre l'objet du document
%%  - faire les footers avec l'adresse et tels de SysFera
%%  - enlever un max de paquets docbook


%%%%%%%%%%%%%%%%%%%%%%%%%%%%%%%%%%%%%%%%%%%%%%%%%%%%%%%%%%%%%%%%%%%%%%
%                            CONFIGURATION                           %
%%%%%%%%%%%%%%%%%%%%%%%%%%%%%%%%%%%%%%%%%%%%%%%%%%%%%%%%%%%%%%%%%%%%%%
% Use the following macros to configure your document
%%%%%%%%%%%%%%%%%%%%%%%%%%%%%%%%%%%%%%%%%%%%%%%%%%%%%%%%%%%%%%%%%%%%%%
%%%% OPTION
%% Change language: fr/en
%%  - for French: use french in babel, and fr in \setupsysferalocale
%%  - for English: use english in babel, and en in \setupsysferalocale
\usepackage[english]{babel}
\setupsysferalocale{en}

%%%% OPTION
%% Title and author of the document
\title{New approach of the forwarder architecture}
\author{K. COULOMB}

%%%% OPTION
%% Document reference
%% Use command \SFdocumentreference to set the document reference.
%% Latter on, you can use the \SFthisdocument macro to retrieve
%% this reference.
\SFdocumentreference{Forwarder Architechture}
\SFprojectname{DIETv3}
\SFprojectleader{B. DEPARDON}
%\SFclient{God}


%%%% OPTION
%% Release information. If the argument is not empty, then a box with
%% the content of the argument will be visible at the top of the document
\SFreleaseinfo{} % will show "Travail en cours" at the
                                 % top of the page
%\SFreleaseinfo{} % won't show anything

%%%% OPTION
%% Draft watermark. You can also show a grey watermak on all pages of
%% your document using the following command.
%\showwatermark{DRAFT}

%%%% OPTION
%% Collaborators:
%% You can redefine the Indexation of the document using the following command:
% \renewcommand{\SFindexation}{
%   \begin{SFindtable}
%     \SFinditem{\writtenby}{Author 1}{06/06/2011}
%     \SFinditem{\correctedby}{Author 2}{22/07/2011}
%     \SFinditem{\validatedby}{Anonymous guy}{\date}
%   \end{SFindtable}
% }
% \renewcommand{\SFindexation}{} % disable this table


%%%% OPTION
%% Revision History Table:
%% Add a new \SFrevitem entry for adding a new entry in the revision
%% history table. Revision versions are added automatically.
\renewcommand{\SFrevhistory}{
\begin{SFrevtable}
  \SFrevitem{02/08/2011}{First draft of the document}{K. COULOMB}
  \SFrevitem{16/11/2011}{New version with functionnal POC}{K. COULOMB}
\end{SFrevtable}
}
% \renewcommand{\SFrevhistory}{} % disable this table


%%%% OPTION
%% References Table
%% Add a new \SFrefitem entry for adding a new entry in the list of
%% reference documents.
\renewcommand{\SFreferenceTable}{
\begin{SFreftable}
\end{SFreftable}
}
\renewcommand{\SFreferenceTable}{} % disable this table

%%%% OPTION
%% Authorization Table
%% Add a new \SFauthviewitem entry for adding a new entry in the list of
%% authorized users.
\renewcommand{\SFauthview}{
\begin{SFauthviewtable}
  \SFauthviewitem{SysFera}{tech@sysfera.com}
  \SFauthviewitem{Universit� Amiens}{Gael LE MAHEC}
\end{SFauthviewtable}
}
% \renewcommand{\SFauthview}{} % disable this table
%%%%%%%%%%%%%%%%%%%%%%%%%%%%%%%%%%%%%%%%%%%%%%%%%%%%%%%%%%%%%%%%%%%%%%
%                           /CONFIGURATION                           %
%%%%%%%%%%%%%%%%%%%%%%%%%%%%%%%%%%%%%%%%%%%%%%%%%%%%%%%%%%%%%%%%%%%%%%


\makeindex
\makeglossary


\begin{document}

\frontmatter % do not disable, this is used for page numbering
\maketitle % do not disable, otherwise you won't have any title
%%%% OPTION
%\tableofcontents % comment to disable the table of contents
\mainmatter % do not disable, this is used for page numbering


%%%%%%%%%%%%%%%%%%%%%%%%%%%%%%%%%%%%%%%%%%%%%%%%%%%%%%%%%%%%%%%%%%%%%%
%              Write your document below this comment                %
%%%%%%%%%%%%%%%%%%%%%%%%%%%%%%%%%%%%%%%%%%%%%%%%%%%%%%%%%%%%%%%%%%%%%%

\chapter*{Introduction}

\section*{Overview of the situation}
There currently is a forwarder system that is used in DIET, in Dagda and in the log service. Althought the core functions are the same, there are different programs and lots of code is duplicated to realize the same things. \\
The aim of this document is to present a new architecture that will fix these problems and offer better evolution perspectives.
\section*{Key problems}
\begin{itemize}
\item Code duplication
\item Process duplication
\item Non reuse of objects
\item Main critic point : the \textbf{resolve} function
\item The peer call of specific objects (IDL functions calls)
\end{itemize}

\chapter*{The first try}

\section*{Overview}
The diagram \ref{component} shows on a single domain a plugin decomposition of the solution. A plugin will contain the IDL, the generated subclasses (headers and stubs), their implementation and the implementation of a connector. This connector will inherit from the Connector root class that will be defined with the ORBMgr and not in a plugin. \\
The plugin's connector will subscribe in the ORBMgr, that will keep a reference on him, and use public getter function to get the data needed to resolve.
Hence there will be only one ORBMgr in a process, this means only one ssh tunnel is needed. Indeed, there won't be 3 forwarders (a diet, a log, a dagda) but a generic forwarder (= a simple ssh tunnel, with possible feautures such as a cache). \\
The (Log/Diet)Forwarder classes will lose their SSH part, and a new GenericForwarder class will be used. It will contain all the SSH aspects of the current (Log/Diet)Forwarder classes. This class will be used by the genericForwarder executable.

\section*{The changes needed}
This is a list of identified changes needed
\begin{itemize}
\item Add a list of connectors to the ORBMgr
\item Rewrite resolve using the connectors
\item Create and implement all the connector classes
\item Add the 'communication' methods in the connectors
\item Rewrite the (Log/Diet)Forwarder classes to use the connector
\item Rewrite the forwarders in a generic one
\item Write a generic (Log/Diet)Forwarder class
\item Remove the CORBA aspects from the (Log/Diet)Forwarder classes
\end{itemize}

\section*{Failure}
This solution was a total failure. The problem was that it was impossible to resolve
a CORBA object throught a generic forwarder with the IDL. Moreover, the cast from
a generic forwarder to a forwarder implementation was impossible because the real object
was generic so it could not be cast. This caused the real solution to be less ambitious,
the IDL (and their implementation) are needed in the libdadiCORBA.


\section*{Figures}
\subsection*{Components}
\begin{figure}[h!]
\includegraphics[scale=0.7,angle=90]{fig/component}
\caption{Diagram showing the various component that will be used}
\label{component}
\end{figure}

\subsection*{The class diagram}
\begin{figure}[h!]
\includegraphics[scale=0.65,angle=90]{fig/class}
\caption{Diagram showing the various classes that will be added}
\label{class}
\end{figure}

\subsection*{The deployment diagram}
\begin{figure}[h!]
\includegraphics[scale=0.65, angle=90]{fig/deploiement}
\caption{Diagram showing an example of deployment}
\label{deployment}
\end{figure}

\subsection*{The sequence diagram for resolve}
\begin{figure}[h!]
\includegraphics[scale=0.55, angle=90]{fig/sequence}
\caption{Diagram showing the sequence diagram on the resolve function}
\label{sequence}
\end{figure}


\chapter*{Other possibilities}
\section*{The CORBA interceptor}
Haikel presented an idea based on CORBA interceptor. \\
The CORBA norm defines the possibility that the users can make interceptor in some specific points. 
The idea would be to use interceptors to route to the right component that possesses the method required by the object.

\section*{Share the ORBMgr, not the tunnels}
Similar to the presented idea, but the forwarder with the main function are kept separated.

\chapter*{The real solution}
\section*{The degraded version}
Because the version objective is impossible, the implemented solution is a degraded version of the previous one.
The specificities of the solution are the following ones:
\begin{itemize}
\item One ORBMgr to rule them all
\item One tunnel to find them 
\item One lib to bring them and in the linker bind them
\end{itemize}
These modifications imply to have all the IDL and the forwarder class implementations in the generic library.

\section*{Implementation}
The POC implementation is based on 2 parts, the core library and the log implementation.

\subsection*{The core (in the CORBA directory)}
The core contains various main elements, it represents the generic lib.
\begin{itemize}
\item The IDL repository that contains all the CORBA IDL
\item The tools and utils repositories that contain common classes (like the root directory)
\item The monitor repository that contains the LogCentral implementation
\item The forwarder repository that contains the CORBAForwarder class implementation
\item The impl repository that contains the forwarder implementation of the DIET IDL stubs
\item The log repository that contains the CORBAForwarder class implementation for the log functions
\item The diet repository that contains the CORBAForwarder class implementation for the diet functions
\item The Cmake repository that contains the files for cmake to find the dependencies
\item The Fwdr.cc file that is the source for the binary forwarder program
\end{itemize}

\subsection*{The log}
The log is a project example using the core lib to use the forwarders. It contains the following elements
\begin{itemize}
\item The Cmake repository that contains the files for cmake to find the dependencies
\item The example repository that contains a tool and a component example
\item The libraries repository that contains a lib (may be deprecated but currently kept)
\item Log specific files
\end{itemize}

\section*{Function}
This POC works the same way the previous forwarder were working, the only difference is that the ORBMgr and
the tunnel are reunited. But all the basis mechanisms are kept. The ORBMgr was practically not touched 
(only the resolveObject function was modified to be able to return different kind of objects). And the
CORBAForwarder class is nothing but a merge of the DIETForwarder class and the LogForwarder class.


\section*{Known issues}
\begin{itemize}
\item Modify an IDL means recompile all the depending projects
\item Tools and Components are the only item really tested
\item POC based on both old and dirty codes
\end{itemize}

\section*{Industrialization}
The following section will present the different element to transform the POC into a real functionnal library. 
These transformation will be for both DIET and the log service current repository.

\subsection*{Modification}
\begin{itemize}
\item Extract and install the following headers 
\begin{itemize}
\item ComponentList.hh
\item FilterManagerInterface.hh
\item FullLinkedList.hh
\item FullLinkedList.cc (included in FullLinkedList.hh)
\item LocalTime.hh
\item LogOptions.hh
\item ORBMgr.hh
\item ReadConfig.hh
\item response.hh
\item StateManager.hh
\item common\_types.hh
\item CorbaForwarder.hh
\item Forwarder.hh
\item LogTypes.hh
\item TimeBuffer.hh
\item ToolList.hh
\item Options.hh ? (not sure)
\end{itemize}
\item Install the generated stubs header
\begin{itemize}
\item Agent(Fwdr).hh
\item Callback(Fwdr).hh
\item LocalAgent(Fwdr).hh
\item LogComponent(Fwdr).hh
\item LogTool(Fwdr).hh
\item MasterAgent(Fwdr).hh
\item SeD(Fwdr).hh
\end{itemize}
\item Extract forwarder implementation from basic stubs implementations for each kind of IDL
\item Extract internal utilitaries files
\begin{itemize}
\item debug.(cc/hh)
\item DIET\_data.hh)
\item Options.(cc/hh)
\item FullLinkedList.(cc/hh)
\item ORBTools.(cc/hh)
\item SSHTunnels.(cc/hh)
\item LocalTime.(cc/hh)
\item LogOptions.(cc/hh)
\item ReadConfig.(cc/hh)
\item StateManager.(cc/hh)
\item TimeBuffer.(cc/hh)
\item ToolList.hh
\item dietObject.cc
\item The LogCentralImplementation for tools and components
\end{itemize}
\item R�cup�rer le findOmniORB
\item R�cup�rer et les merger dans une m�me classe CORBAForwarder les classes DIETForwarder et LogForwarder
\item Clean the code/includes (lots of dirty things are in the code, starting from an already cleaned version by Haikel is great but may not be enought)
\end{itemize}

\section*{Tests}
Various tests have been made between graal and a local laptop. The biggest test was between karadoc, ukalerk and graal. The figure \ref{test} presents the deployment. There was 3 omninames (one per machine), 2 forwarders (one between karadoc and ukalerk, and one between graal and ukalerk). The LogCentral was on ukalerk, the tool was on karadoc, the component was on graal. The tool displayed what the component sent.

\begin{figure}[h!]
\includegraphics[scale=0.4]{fig/test_dadiCORBA.png}
\caption{Diagram showing the deployement for the test}
\label{test}
\end{figure}

\end{document}
